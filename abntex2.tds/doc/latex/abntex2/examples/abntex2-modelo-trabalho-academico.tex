%% USPSC-modelo.tex
% ---------------------------------------------------------------
% USPSC: Modelo de Trabalho Academico (tese de doutorado, dissertacao de
% mestrado e trabalhos monograficos em geral) em conformidade com
% ABNT NBR 14724:2011: Informacao e documentacao - Trabalhos academicos -
% Apresentacao
%----------------------------------------------------------------
%% Esta é uma customização do abntex2-modelo-trabalho-academico.tex de v-1.9.5 laurocesar
%% para as Unidades do Campus USP de São Carlos:
%% EESC - Escola de Engenharia de São Carlos
%% IAU - Instituto de Arquitetura e Urbanismo
%% ICMC - Instituto de Ciências Matemáticas e de Computação
%% IFSC - Instituto de Física de São Carlos
%% IQSC - Instituto de Química de São Carlos
%%
%% Este trabalho utiliza a classe USPSC.cls que é mantida pela seguinte equipe:
%%
%% Coordenação e Programação:
%%   - Marilza Aparecida Rodrigues Tognetti - marilza@sc.usp.br (PUSP-SC)
%%   - Ana Paula Aparecida Calabrez - aninha@sc.usp.br (PUSP-SC)
%% Normalização:
%%   - Brianda de Oliveira Ordonho Sigolo - brianda@usp.br (IAU)
%%   - Eduardo Graziosi Silva - edu.gs@sc.usp.br (EESC)
%%   - Eliana de Cássia Aquareli Cordeiro - eliana@iqsc.usp.br (IQSC)
%%   - Flávia Helena Cassin - cassinp@sc.usp.br (EESC)
%%   - Maria Cristina Cavarette Dziabas - mcdziaba@ifsc.usp.br (IFSC)
%%   - Regina Célia Vidal Medeiros - rcvmat@icmc.usp.br (ICMC)
%%
%% O USPSC-modelo.tex e USPSC-TCC-modelo.tex utilizam diversos arquivos relacionado em
%% 2.1 Pacote USPSC: Classe USPSC e modelos de trabalhos acadêmicos	do Tutorial do Pascote
%%  USPSC para modelos de trabalhos de acadêmicos em LaTeX - versão 3.1


%----------------------------------------------------------------
%% Sobre a classe abntex2.cls:
%% abntex2.cls, v-1.9.5 laurocesar
%% Copyright 2012-2015 by abnTeX2 group at https://www.abntex.net.br/
%%
%----------------------------------------------------------------

\documentclass[
% -- opções da classe memoir --
12pt,		% tamanho da fonte
openright,	% capítulos começam em pág ímpar (insere página vazia caso preciso)
twoside,  % para impressão em anverso (frente) e verso. Oposto a oneside - Nota: utilizar \imprimirfolhaderosto*
%oneside, % para impressão em páginas separadas (somente anverso) -  Nota: utilizar \imprimirfolhaderosto
% inclua uma % antes do comando twoside e exclua a % antes do oneside
a4paper,			% tamanho do papel.
% -- opções da classe abntex2 --
chapter=TITLE,		% títulos de capítulos convertidos em letras maiúsculas
% -- opções do pacote babel --
english,			% idioma adicional para hifenização
french,				% idioma adicional para hifenização
spanish,			% idioma adicional para hifenização
brazil				% o último idioma é o principal do documento
% {USPSC-classe/USPSC} configura o cabeçalho contendo apenas o número da página
]{USPSC-classe/USPSC}
%]{USPSC-classe/USPSC1}
% Inclua % antes de ]{USPSC-classe/USPSC} e retire a % antes de %]{USPSC-classe/USPSC1} para utilizar o
% cabeçalho diferenciado para as páginas pares e ímpares:
%- páginas ímpares: com seções ou subseções e o número da página
%- páginas pares: com o número da página e o título do capítulo
% ---
% ---
% Pacotes básicos - Fundamentais
% ---
\usepackage[T1]{fontenc}		% Seleção de códigos de fonte.
\usepackage[utf8]{inputenc}		% Codificação do documento (conversão automática dos acentos)
\usepackage{lmodern}			% Usa a fonte Latin Modern
% Para utilizar a fonte Times New Roman, inclua uma % no início do comando acima  "\usepackage{lmodern}"
% Abaixo, tire a % antes do comando  \usepackage{times}
%\usepackage{times}		    	% Usa a fonte Times New Roman
% Para usar a fonte , lembre-se de tirar a % do comando %\renewcommand{\ABNTEXchapterfont}{\rmfamily}, localizado mais abaixo, logo após "Outras opções para nota de rodapé no Sistema Numérico"
\usepackage{lastpage}			% Usado pela Ficha catalográfica
\usepackage{indentfirst}		% Indenta o primeiro parágrafo de cada seção.
\usepackage{color}				% Controle das cores
\usepackage{graphicx}			% Inclusão de gráficos
\usepackage{float} 				% Fixa tabelas e figuras no local exato
\usepackage{chemfig,chemmacros} % Para escrever reações químicas
\usepackage{tikz}				% Para escrever reações químicas e outros
\usetikzlibrary{positioning}
\usepackage{microtype} 			% para melhorias de justificação
\usepackage{pdfpages}
\usepackage{makeidx}            % para gerar índice remissivo
\usepackage{hyphenat}          % Pacote para retirar a hifenizacao do texto
\usepackage[absolute]{textpos} % Pacote permite o posicionamento do texto
\usepackage{eso-pic}           % Pacote para incluir imagem de fundo
\usepackage{makebox}           % Pacote para criar caixa de texto
% ---

% ---
% Pacotes de citações
% Citações padrão ABNT
% ---
% Sistemas de chamada: autor-data ou numérico.
% Sistema autor-data
\usepackage[alf, abnt-emphasize=bf, abnt-thesis-year=both, abnt-repeated-author-omit=no, abnt-last-names=abnt, abnt-etal-cite, abnt-etal-list=3, abnt-etal-text=it, abnt-and-type=e, abnt-doi=doi, abnt-url-package=none, abnt-verbatim-entry=no]{abntex2cite}
\bibliographystyle{USPSC-classe/abntex2-alf-USPSC}
% Se o idioma for o inglês, inclua % no comando acima e exclua o % do comando abaixo
%\bibliographystyle{USPSC-classe/abntex2-alfeng-USPSC}

% Para o IQSC, que indica todos os autores nas referências, incluir % no início dos comandos acima e retirar a % dos comandos abaixo
%\usepackage[alf, abnt-emphasize=bf, abnt-thesis-year=both, abnt-repeated-author-omit=no, abnt-last-names=abnt, abnt-etal-cite, abnt-etal-list=0, abnt-etal-text=it, abnt-and-type=e, abnt-doi=doi, abnt-url-package=none, abnt-verbatim-entry=no]{abntex2cite}
%\bibliographystyle{USPSC-classe/abntex2-alf-USPSC}
% Se o idioma for o inglês, exclua % no comando acima ou do comando abaixo
%\bibliographystyle{USPSC-classe/abntex2-alfeng-USPSC}

% Sistema Numérico
% Para citações numéricas, sistema adotado pelo IFSC, incluir % no início dos comandos acima e retirar a % dos comandos abaixo
%\usepackage{cite}             % agrupa citações numéricas consecutivas
%\usepackage[num, abnt-emphasize=bf, abnt-thesis-year=both, abnt-repeated-author-omit=no, abnt-last-names=abnt, abnt-etal-cite, abnt-etal-list=3, abnt-etal-text=it, abnt-and-type=e, abnt-doi=doi, abnt-url-package=none, abnt-verbatim-entry=no]{abntex2cite}
%\bibliographystyle{USPSC-classe/abntex2-num-USPSC}
% Se o idioma for o inglês, exclua % no comando acima ou do comando abaixo
%\bibliographystyle{USPSC-classe/abntex2-numeng-USPSC}

% Complementarmente, verifique as instruções abaixo sobre os Pacotes de Nota de rodapé
% ---
% Pacotes de Nota de rodapé
% Configurações de nota de rodapé

% O presente modelo adota o formato numérico para as notas de rodapés quando utiliza o sistema de chamada autor-data para citações e referências. Para utilizar o sistema de chamada numérico para citações e referências, habilitar um dos comandos abaixo.
% Há diversa opções para nota de rodapé no Sistema Numérico.  Para o IFSC, habilitade o comando abaixo.

%\renewcommand{\thefootnote}{\fnsymbol{footnote}}  %Comando para inserção de símbolos em nota de rodapé

% Outras opções para nota de rodapé no Sistema Numérico:
%\renewcommand{\thefootnote}{\alph{footnote}}      %Comando para inserção de letras minúscula em nota de rodapé
%\renewcommand{\thefootnote}{\Alph{footnote}}      %Comando para inserção de letras maiúscula em nota de rodapé
%\renewcommand{\thefootnote}{\roman{footnote}}     %Comando para inserção de números romanos minúsculos  em nota de rodapé
%\renewcommand{\thefootnote}{\Roman{footnote}}     %Comando para inserção de números romanos minúsculos  em nota de rodapé

\renewcommand{\footnotesize}{\small} %Comando para diminuir a fonte das notas de rodapé
%Para utilizar a fonte Times New Roman, inclua retire % do início do comando abaixo
%\renewcommand{\ABNTEXchapterfont}{\rmfamily}

% ---
% Pacote para agrupar a citação numérica consecutiva
% Quando for adotado o Sistema Numérico, a exemplo do IFSC, habilite
% o pacote cite abaixo retirando a porcentagem antes do comando abaixo
%\usepackage[superscript]{cite}

% ---
% Pacotes adicionais, usados apenas no âmbito do Modelo Canônico do abnteX2
% ---
\usepackage{lipsum}				% para geração de dummy text
% ---

% pacotes de tabelas
\usepackage{multicol}	% Suporte a mesclagens em colunas
\usepackage{multirow}	% Suporte a mesclagens em linhas
\usepackage{longtable}	% Tabelas com várias páginas
\usepackage{threeparttablex}    % notas no longtable
\usepackage{array}

% ----
% Compatibilização com a ABNT NBR 6023:2018
% Para tirar <> da URL
%\DeclareFieldFormat{url}{\bibstring{urlfrom}\addcolon\addspace\url{#1}}
\usepackage{USPSC-classe/ABNT6023-2018}
% As demais compatibilizações estão nos arquivos abntex2-alf-USPSC.bst,abntex2-alfeng-USPSC.bst, abntex2-num-USPSC.bst e abntex2-numeng-USPSC.bst, dependendo do idioma do textos e se o sistemas de chamada for autor-data ou numérico, conforme explicitado acima.
% ----

% ---
% DADOS INICIAIS - Define sigla com título, área de concentração e opção do Programa
% Consulte a tabela referente aos Programas, áreas e opções de sua unidade contante do
% arquivo USPSC-Siglas estabelecidas para os Programas de Pós-Graduação por Unidade.xlsx
% ou nos APÊNDICES A-F
\siglaunidade{XXXX}
\programa{XXXX}
% Os demais dados deverão ser fornecidos no arquivo USPSC-pre-textual-UUUU ou USPSC-TCC-pre-textual-UUUU, onde UUUU é a sigla da Unidade.
% Exemplo:USPSC-pre-textual-IFSC.tex
% ---
% Configurações de aparência do PDF final
% alterando o aspecto da cor azul
\definecolor{blue}{RGB}{41,5,195}

% informações do PDF
\makeatletter
\hypersetup{
	%pagebackref=true,
	pdftitle={\@title},
	pdfauthor={\@author},
	pdfsubject={\imprimirpreambulo},
	pdfcreator={LaTeX with abnTeX2},
	pdfkeywords={abnt}{latex}{abntex}{USPSC}{trabalho acadêmico},
	colorlinks=true,       		% false: boxed links; true: colored links
	linkcolor=black,          	% color of internal links
	citecolor=black,        		% color of links to bibliography
	filecolor=black,      		% color of file links
	urlcolor=black,
	%Para habilitar as cores dos links, retire a % antes dos comandos abaixo e inclua a % antes das 4 linhas de comando acima
	%linkcolor=blue,            	% color of internal links
	%citecolor=blue,        		% color of links to bibliography
	%filecolor=magenta,      		% color of file links
	%urlcolor=blue,
	bookmarksdepth=4
}
\makeatother
% ---
% ---
% Espaçamentos entre linhas e parágrafos
% ---

% O tamanho do parágrafo é dado por:
\setlength{\parindent}{1.3cm}

% Controle do espaçamento entre um parágrafo e outro:
\setlength{\parskip}{0.2cm}  % tente também \onelineskip

% ---
% compila o sumário e índice
\makeindex
% ---

% ----
% Início do documento
% ----
\begin{document}

% Seleciona o idioma do documento (conforme pacotes do babel)
\selectlanguage{brazil}
% Se o idioma do texto for inglês, inclua uma % antes do
%      comando \selectlanguage{brazil} e
%      retire a % antes do comando abaixo
%\selectlanguage{english}

% Retira espaço extra obsoleto entre as frases.
\frenchspacing

% --- Formatação dos Títulos
\renewcommand{\ABNTEXchapterfontsize}{\fontsize{12}{12}\bfseries}
\renewcommand{\ABNTEXsectionfontsize}{\fontsize{12}{12}\bfseries}
\renewcommand{\ABNTEXsubsectionfontsize}{\fontsize{12}{12}\normalfont}
\renewcommand{\ABNTEXsubsubsectionfontsize}{\fontsize{12}{12}\normalfont}
\renewcommand{\ABNTEXsubsubsubsectionfontsize}{\fontsize{12}{12}\normalfont}


% ----------------------------------------------------------
% ELEMENTOS PRÉ-TEXTUAIS
% ----------------------------------------------------------
% ---
% Capa
% ---
\imprimircapa
% ---
% Folha de rosto
% (o * indica impressão em anverso (frente) e verso )
% ---
\imprimirfolhaderosto*
%\imprimirfolhaderosto
% ---
% ---
% Inserir a ficha catalográfica em pdf
% ---
% A biblioteca da sua Unidade lhe fornecerá um PDF com a ficha
% catalográfica definitiva.
% Quando estiver com o documento, salve-o como PDF no diretório
% do seu projeto como fichacatalografica.pdf e inclua o arquivo
% utilizando o comando abaixo:

\includepdf{USPSC-TA-PreTextual/USPSC-fichacatalografica.pdf}

% Se você optar por elaborar a ficha catalográfica, deverá
% incluir uma % antes da linha % antes
% do comando \include{USPSC-TA-PreTextual/USPSC-fichacatalografica}
% e retirar o % do comando abaixo
%\include{USPSC-TA-PreTextual/USPSC-fichacatalografica}
% As informações que compõem a ficha catalográfica estão
% definidas no arquivo USPSC-pre-textual-UUUU.tex
% ---

% ---
% Folha de rosto adicional
% Para imprimir a folha de rosto adicional, exigida por algumas Unidades, a exemplo do ICMC,
% retire a % antes do comando abaixo

%\imprimirfolhaderostoadic

% ---
% ---
% Inserir errata
% ---

\include{USPSC-TA-PreTextual/USPSC-Errata}

% ---

% ---
% Inserir folha de aprovação
% ---

% A Folha de aprovação é um elemento obrigatório da NBR 4724/2011 (seção 4.2.1.3).
% Após a defesa/aprovação do trabalho, gere o arquivo folhadeaprovacao.pdf da página assinada pela banca
% e iclua o arquivo utilizando o comando abaixo:
\includepdf{USPSC-TA-PreTextual/USPSC-folhadeaprovacao.pdf}
% Alternativa para a Folha de Aprovação:
% Se for a sua opção elaborar uma folha de aprovação, insira uma % antes do comando acima que inclui o arquivo folhadeaprovacao.pdf,
% tire o % do comando abaixo e altere o arquivo folhadeaprovacao.tex conforme suas necessidades
%\include{folhadeaprovacao}
\includepdf{USPSC-TA-PreTextual/USPSC-PaginaEmBranco.pdf}

% ---
% Dedicatória
% ---
\include{USPSC-TA-PreTextual/USPSC-Dedicatoria}
% ---

% ---
% Agradecimentos
% ---
\include{USPSC-TA-PreTextual/USPSC-Agradecimentos}
% ---

% ---
% Epígrafe
% ---
\include{USPSC-TA-PreTextual/USPSC-Epigrafe}
% ---

% A T E N Ç Ã O
% Se o idioma do texto for em inglês, o abstract deve preceder o resumo
% resumo em português
%
% Resumo
% ---
\include{USPSC-TA-PreTextual/USPSC-Resumo}
% ---

% Abstract
% ---
\include{USPSC-TA-PreTextual/USPSC-Abstract}
% ---

% ---
% inserir lista de figurass
% ---
\pdfbookmark[0]{\listfigurename}{lof}
\listoffigures*
\cleardoublepage
% ---

% ---
% inserir lista de tabelas
% ---
\pdfbookmark[0]{\listtablename}{lot}
\listoftables*
\cleardoublepage
% ---

% ---
% inserir lista de quadros
% ---
\pdfbookmark[0]{\listofquadroname}{loq}
\listofquadro*
\cleardoublepage
% ---

% ---
% inserir lista de abreviaturas e siglas
% ---
\include{USPSC-TA-PreTextual/USPSC-AbreviaturasSiglas}
% ---

% ---
% inserir lista de símbolos
% ---
\include{USPSC-TA-PreTextual/USPSC-Simbolos}
% ---
% ---
% inserir o sumario
% ---
\pdfbookmark[0]{\contentsname}{toc}
\tableofcontents*
\cleardoublepage
% ---
% ----------------------------------------------------------
% ELEMENTOS TEXTUAIS
% ----------------------------------------------------------
\textual
% Os capítulos são inseridos como arquivos externos

% Capítulo 1 - Introdução
% ---
\include{USPSC-TA-Textual/USPSC-Cap1-Introducao}
% ---

% ---
% Capítulo 2
% ---
\include{USPSC-TA-Textual/USPSC-Cap2-Desenvolvimento}

% Capítulo 3 - Conclusão
% ---
\include{USPSC-TA-Textual/USPSC-Cap3-Conclusao}
% ---

% ----------------------------------------------------------
% ELEMENTOS PÓS-TEXTUAIS
% ----------------------------------------------------------
\postextual
% ----------------------------------------------------------

% -----------------------------------------------------------
% Referências bibliográficas
% ----------------------------------------------------------
\bibliography{USPSC-bib/USPSC-modelo-references}


% ----------------------------------------------------------
% Glossário
% ----------------------------------------------------------
%
% Consulte o manual da classe abntex2 para orientações sobre o glossário.
%
%\glossary

% ----------------------------------------------------------
% Apêndices
% ----------------------------------------------------------
\include{USPSC-TA-PosTextual/USPSC-Apendices}

% ----------------------------------------------------------
% Anexos
% ----------------------------------------------------------
\include{USPSC-TA-PosTextual/USPSC-Anexos}

%---------------------------------------------------------------------
% INDICE REMISSIVO
%--------------------------------------------------------------------
\include{USPSC-TA-PosTextual/USPSC-IndicesRemissivos}

%---------------------------------------------------------------------

\end{document}
